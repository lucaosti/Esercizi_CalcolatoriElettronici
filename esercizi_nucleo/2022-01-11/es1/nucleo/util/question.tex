Definiamo una ``memory-area'' come una zona della memoria privata di un processo,
accessibile in lettura/scrittura a partire dal primo indirizzo della parte utente/privata.
I processi non hanno inizialmente una memory-area, ma possono crearne una tramite
la primitiva \verb|macreate(dim)|, dove \verb|dim| \`e la dimensione in pagine.

Un processo sorgente pu\`o inviare una {\em copia} della propria memory-area ad un altro
processo destinario, con identificatore \verb|pid|, usando la primitiva \verb|macopy(pid)|,
purch\'e il destinatario non ne possegga gi\`a un'altra. 
Pi\`u processi possono ricevere una copia della stessa memory-area, sia direttamente dal
processo che l'aveva creata, sia indirettamente da un processo che l'aveva a sua volta
ricevuta.

Per ottimizzare la copia utilizziamo il meccanismo del {\em copy-on-write}:
la \verb|macopy()| fa inizialmente accedere entrambi i processi (sorgente e destinario)
agli stessi frame della memoria fisica, ma in sola lettura. La copia verr\`a eseguita solo quando,
successivamente, uno dei processi tenta di scrivervi. Questo permette di evitare le copie
se i processi si limitano a leggere.

Visto che la stessa memory-area pu\`o essere copiata pi\`u volte, ogni frame
di una memory-area pu\`o essere condiviso tra pi\`u processi. Aggiungiamo dunque ai descrittori
dei frame (\verb|des_frame|) un campo \verb|nma| destinato a contare il numero di processi che condividono
quel frame. Quando la memory-area viene creata \verb|nma| viene posto a $1$ per tutti
i frame che la compongono; quando la memory-area viene copiata
tutti i suoi \verb|nma| vengono incrementati.

Come detto, ciascun frame verr\`a effettivamente
copiato solo quando, e se, uno dei processi  che lo condividono tenter\`a di accedervi in scrittura.
Pi\`u precisamente, consideriamo un frame $f$ condiviso tra i processi $P_1$, $P_2$, \dots, $P_n$
(quindi con $\verb|nma| = n$). Se un qualsiasi processo $P_i$, con $1\le i \le n$, tenta di
accedere in scrittura a un indirizzo $v$ mappato su $f$, la MMU causer\`a un page fault.
La routine di gestione, riconosciuta la causa del page fault, allocher\`a un nuovo frame $f'$ e
vi copier\`a il contenuto di $f$, decrementandone \verb|nma| e cambiando la traduzione di $v$
in modo che ora punti a $f'$ e la scrittura sia abililitata. A questo punto il processo $P_i$
potr\`a ripetere l'accesso.

Per realizzare il meccanismo appena descritto aggiungiamo il campo \verb|masize| ai descrittori
di processo. Il campo contiene la dimensione in pagine della memory-area del processo; vale 0 se il processo
non ha ancora creato o ricevuto una memory-area.

Aggingiamo inoltre le seguenti primitive (abortiscono il processo in caso di errore):

\begin{itemize}
  \item \verb|void* macreate(natq size)| (realizzata in parte): crea una memory-area di \verb|size| pagine.
    \`E un errore se \verb|size| \`e zero o maggiore di \verb|MAX_MA_PAGES|, oppure se il
    processo possiede gi\`a una memory-area. Restituisce l'indirizzo della memory-area, o \verb|nullptr|
    se non \`e stato possibile crearla.
  \item \verb|bool macopy(natq pid)| (realizzata in parte): invia al processo \verb|pid| una
    copia della memory-area del processo chiamante, usando il meccanismo del copy-on-write.
    \`E un errore se il processo chiamante non possiede una memory-area o se \verb|pid|
    supera \verb|MAX_PROC_ID|\@.
    Restituisce \verb|false| se il processo destinatario non esiste o possiede gi\`a una memory area,
    oppure se non \`e stato possibile completare l'operazione (memoria esaurita); restituisce \verb|true|
    altrimenti.
\end{itemize}

Modificare il file \verb|sistema.cpp| in modo da specificare la parti mancanti. Attenzione a deallocare
le memory-area correttamente quando un processo termina.
