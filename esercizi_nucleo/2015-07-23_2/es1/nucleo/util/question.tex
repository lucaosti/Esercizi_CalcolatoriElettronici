Prevediamo che un processo possa creare delle zone di memoria, dette 
\verb|shmem|, ciascuna con
un identificatore unico. I processi che vogliono accedere ad una \verb|shmem|
devono aggiungerla al proprio spazio di indirizzamento, specificandone l'identificatore.
Una volta aggiunta, la \verb|shmem| sar\`a disponibile contiguamente all'interno della parte
utente/condivisa dello spazio di indirizzamento del processo. Un processo pu\`o aggiungere
pi\`u \verb|shmem| al proprio spazio e le diverse \verb|shmem| non devono sovrapporsi.
Non \`e importante che i processi che condividono una stessa \verb|shmem| la vedano tutti
allo stesso indirizzo.
In qualunque momento, un processo pu\`o eliminare dal proprio spazio di indirizzamento 
una \verb|shmem| precedentemente aggiunta.

Per descrivere una \verb|shmem| aggiungiamo al nucleo la seguente struttura dati:

\begin{verbatim}
    struct des_shmem {
        natl npag;
        natl first_frame_number;
    };
\end{verbatim}

Il campo \verb|npag| contiene la dimensione (in pagine) della \verb|shmem|.
Tutte i frame che contengono la \verb|shmem|, nell'ordine in cui
devono comparire nella memoria di tutti i processi che la condividono,
sono mantenuti in una lista; il numero del frame in testa alla lista \`e contenuto nel campo
\verb|first_frame_number|. Ogni frame
punta al successivo tramite un nuovo campo \verb|natl next_shmem| che abbiamo
aggiunto ai descrittori di frame.

Inoltre, aggiungiamo i seguenti campi ai descrittori di processo:

\begin{verbatim}
    vaddr avail_addr;
    des_attached *att;
\end{verbatim}

Il campo \verb|avail_addr| contiene il primo indirizzo libero nella parte utente/condivisa del processo.
Tutti gli indirizzi da \verb|avail_addr| fino a \verb|fin_utn_c| (escluso) sono disponibili
per contenere zone di memoria condivisa.
Il campo \verb|att| \`e la testa di una lista di elementi di tipo \verb|des_attached|,
il cui scopo \`e di tener traccia di tutte le \verb|shmem| a cui il processo \`e
collegato. Il tipo \verb|des_attached| \`e cos\`i definito:

\begin{verbatim}
    struct des_attached {
        natl id;
        vaddr start;
        des_attached *next;
    }
\end{verbatim}

Il campo \verb|id| \`e l'identificatore di una shmem \verb|shmem|; 
il campo \verb|start| \`e l'indirizzo virtuale (nello spazio di indirizzamento
del processo) a partire dal quale questa \verb|shmem| \`e visibile.
Il campo \verb|next| punta al prossimo elemento della lista.

Aggiungiamo infine le seguenti primitive:
\begin{itemize}
   \item \verb|natl shmem_create(natl npag)| (tipo 0x5c, gi\`a realizzata):
   	Crea una nuova zona di memoria condivisibile tra pi\`u processi, grande \verb|npag| pagine,
	e ne restituisce l'identificatore.
   \item \verb|vaddr shmem_attach(natl id)| (tipo 0x5d, gi\`a realizzata):
   	Permette ad un processo di aggiungere la \verb|shmem| \verb|id| al proprio spazio di indirizzamento
	e ne restituisce l'indirizzo di partenza.
   \item \verb|void shmem_detach(natl id)| (tipo 0x5e, da realizzare):
   	Permette ad un processo di eliminare la \verb|shmem| \verb|id| dal proprio
	spazio di indirizzamento. Abortisce il processo se la \verb|shmem| \verb|id|
	non \`e tra quelle a cui il processo \`e collegato.
\end{itemize}

Modificare i file \verb|sistema.cpp| e \verb|sistema.S| in modo da realizzare le primitive appena descritte.

{\bf ATTENZIONE}: nella \verb|shmem_detach()|, tralasciare la gestione di \verb|avail_addr|.
