Nel sistema base, tutti i processi utente partono con una pila utente della
stessa dimensione, pari a {\tt DIM\_USR\_STACK} (la costante \`e definita in {\tt include/costanti.h}).
Normalmente, per\`o, la maggior parte dei processi
useranno solo una piccola parte della loro pila, e il resto dello spazio sar\`a sprecato.

Per migliorare l'uso della memoria fisica realizziamo il meccanismo dell'{\em estensione automatica dello stack}.
Con questo meccanismo un processo viene creato con una pila utente di dimensioni fisiche inferiori a {\tt DIM\_USR\_STACK}\@. Se, durante
la sua esecuzione, il processo accede a parti della pila che non erano state allocate, il sistema non abortisce il processo,
ma estende automaticamente la sua pila e lo lascia proseguire.

Per realizzare il meccanismo di cui sopra possiamo intercettare i page fault e osservare se per caso
l'indirizzo non tradotto cade all'interno degli indirizzi previsti per la pila utente. In caso
affermativo estendiamo la pila in modo che includa tutti gli indirizzi che vanno dall'indirizzo non tradotto
fino alla base della pila e lasciamo che il processo prosegua la sua esecuzione,
altrimenti ritorniamo alla normale gestione dell'eccezione (che abortir\`a il processo).

Modificare il file {\tt sistema.cpp} per realizzare il meccanismo appena descritto. La pila utente iniziale
dei processi deve essere grande una sola pagina. L'estensione della pila non deve spingersi oltre il minimo
indirizzo necessario a risolvere il page fault. Inoltre, in ogni caso, la dimensione complessiva della pila di ogni processo non deve mai
superare {\tt DIM\_USR\_STACK}\@. Quando si distrugge un processo, ricordarsi di deallocare anche le eventuali risorse
allocate durante le estensioni della pila.
